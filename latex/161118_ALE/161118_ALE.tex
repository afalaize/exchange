\documentclass[11pt,a4paper]{article}
\usepackage[utf8]{inputenc}
\usepackage{amsmath}
\usepackage{amsfonts}
\usepackage{amssymb}
\usepackage{graphicx}
\author{Falaize}
\title{Équation de la chaleur: \\ ALE, Éléments finis, POD et réduction de modèle}
%
\newcommand{\nnodes}{{n_{\labelname{nœuds}}}}
\newcommand{\nx}{{n_{\labelname{X}}}}
\newcommand{\nt}{{n_{\labelname t}}}
\newcommand{\npod}{{n_{\labelname{POD}}}}
\renewcommand{\nt}{{n_{\labelname T}}}
\newcommand{\nc}{{n_{\labelname C}}}
%
\newcommand{\Dtens}{{\mathbf{D}}}
%\newcommand{\Dtens}{{\overline{\overline{\overline{\mathbf{D}}}}}}

\date{\today}
%
%
%
% Bold symbols
\DeclareSymbolFont{bbold}{U}{bbold}{m}{n}
\DeclareSymbolFontAlphabet{\mathbbold}{bbold}

\newcommand{\zeros}{\mathbbold{0}}

\newcommand{\spacename}[1]{#1}
\newcommand{\labelname}[1]{\mathrm{#1}}

\newcommand{\prodscal}[2]{(#1 \vert #2)}
\newcommand{\module}[1]{\vert #1 \vert}
\newcommand{\norme}[1]{\parallel #1 \parallel}

\newcommand{\nthpartial}[3]{\frac{\partial^{#3} #1}{\partial #2^{#3}}}
\renewcommand{\div}{\operatorname{div}}
\newcommand{\grad}{\operatorname{grad}}
\newcommand{\laplacien}{\Delta}
\newcommand{\totalder}[2]{\frac{\D #1}{\D #2}}

\renewcommand{\d}{\operatorname{d}}
\newcommand{\D}{\operatorname{D}}

\newcommand{\T}{\intercal}

\newcommand{\operatorL}{\operatorname{L}}

\newcommand{\NRJ}{E}
\newcommand{\NRJspe}{E_{\labelname{spe}}}
\newcommand{\NRJint}{E_{\labelname{int}}}

\renewcommand{\vec}[1]{\boldsymbol{#1}}
\newcommand{\mat}[1]{\mathbf{#1}}

\newcommand{\RR}{{\mathbb{R}}}
\newcommand{\CC}{{\mathbb{C}}}

\newcommand{\cf}{\textit{c.f.}}
\newcommand{\ie}{\textit{i.e.}}
\newcommand{\Cf}{\textit{C.f.}}
\newcommand{\Ie}{\textit{I.e.}}
%
%
%
\begin{document}
\maketitle
%
%
%
\section{Description ALE}
%
Trois domaines ($\nc$ nombre de composantes):
\begin{description}
\item $\spacename{R}_{\vec x}\subset \RR^{\nc}$ Domaine spatial (espace vide immuable),
\item $\spacename{R}_{\vec X}\subset \RR^{\nc}$ Domaine matériel (attaché à la matière),
\item $\spacename{R}_{\vec \chi}\subset \RR^{\nc}$ Domaine de référence (arbitraire).
\end{description}
%
Le temps est mesuré identiquement dans les trois domaine: $$t\in \spacename T \subseteq \RR$$.
%
Coordonnées spatiales du domaine matériel:
$$ 
\begin{array}{rrcl}
\vec{\varphi}: & (\spacename{R}_{\vec X}, \spacename{T}) & \rightarrow & \left(\spacename{R}_{\vec x}, T\right) \\
& (X, t) &  \mapsto & (x , t)
\end{array}
$$
%
On note $\vec{\varphi}(X, t) = \Big(\vec{\varphi}_{\vec x}(X, t), \vec{\varphi}_{t}(X, t) \Big)=(x , t)$ et
%
$$
\nabla\vec \varphi
=
\left(
\begin{array}{cc}
\partial_{\vec X} \vec{\varphi}_{\vec x} & \vec{v}\\
\zeros^\T & 1
\end{array}
\right)
$$%
%
où on a définit $\vec{v} = \partial_{ t} \vec{\varphi}_{\vec x}  $
%

Coordonnées spatiales du domaine de référence:
$$ 
\begin{array}{rrcl}
\vec{\phi}: & (\spacename{R}_{\vec \chi}, \spacename{T}) & \rightarrow & \left(\spacename{R}_{\vec x}, T\right) \\
& (\chi, t) &  \mapsto & (x , t)
\end{array}
$$
%
On note $\vec{\phi}(\chi, t) = \Big(\vec{\phi}_{\vec x}(\chi, t), \vec{\phi}_{t}(\chi, t) \Big)=(x , t)$ et
%
$$
\nabla\vec \phi
=
\left(
\begin{array}{cc}
\partial_{\vec \chi} \vec{\phi}_{\vec x} & \vec{v}^\star\\
\zeros^\T & 1
\end{array}
\right)
$$%
%
où on a définit $\vec{v}^\star = \partial_{ t} \vec{\phi}_{\vec x}  $
%

Coordonnées de référence du domaine matériel:
$$ 
\begin{array}{rrcl}
\vec{\psi}^{-1}: & (\spacename{R}_{\vec X}, \spacename{T}) & \rightarrow & \left(\spacename{R}_{\vec \chi}, T\right) \\
& (X, t) &  \mapsto & (\chi , t)
\end{array}
$$
%
On note $\vec{\psi}^{-1}(\chi, t) = \Big(\vec{\psi}^{-1}_{\vec \chi}(X, t), \vec{\psi}^{-1}_{t}(X, t) \Big)=(\chi , t)$ et
%
$$
\nabla\vec {\psi}^{-1}
=
\left(
\begin{array}{cc}
\partial_{\vec \chi} \vec{\psi}^{-1}_{\vec x} & \vec{w}\\
\zeros^\T & 1
\end{array}
\right)
$$%
%
où on a définit $\vec{w} = \partial_{ t} \vec{\psi}^{-1}_{\vec \chi}  $.
%
%
%

On a donc $ \varphi  = \phi \circ \psi^{-1}$ et $\nabla \varphi = \nabla\phi \cdot \nabla \psi $, soit
$$ 
\left(
\begin{array}{cc}
\partial_{\vec X} \vec{\varphi}_{\vec x} & \vec{v}\\
\zeros^\T & 1
\end{array}
\right)
 = 
 \left(
\begin{array}{cc}
\partial_{\vec \chi} \vec{\phi}_{\vec x} & \vec{v}^\star\\
\zeros^\T & 1
\end{array}
\right)
\cdot 
\left(
\begin{array}{cc}
\partial_{\vec \chi} \vec{\psi}^{-1}_{\vec x} & \vec{w}\\
\zeros^\T & 1
\end{array}
\right)$$
%
et $\vec{v} = \partial_{\vec \chi} \vec{\phi}_{\vec x} \cdot  \vec{w} + \vec{v}^\star.$
%
Finalement, on définit 
$$ 
\vec c = \vec v - \vec v ^\star = \partial_{\vec \chi} \vec{\phi}_{\vec x} \cdot  \vec{w} .
$$
%
Soit une quantité $q_{\vec x}(x, t)$ mesurée dans le domaine spatial et la même quantité
\begin{description}
\item $q_{\vec X}(X, t)$ mesurée dans le domaine matériel, et
\item  $q_{\vec \chi}(\chi, t)$ mesurée dans le domaine de référence.
\end{description}
%
Alors $q_{\vec X} = q_{\vec x}\circ \vec{\varphi}$ et 
$$ 
\begin{array}{rcl}
\nabla q_{\vec X} &=& \left( \partial _{X}q_{\vec X}, \partial_t q_{\vec X}\right) \\
&=&\nabla q_{\vec x}\cdot \nabla \vec{\varphi}; \\
 \partial _{X}q_{\vec X} &=&  \partial _{x}q_{\vec x}\cdot\partial _{X}\vec \varphi, \\
 \partial _{t}q_{\vec X} &=&  \partial _{x}q_{\vec x}\cdot\vec{v}+\partial_t q_{\vec x}.
\end{array}
$$
%
De même $q_{\vec X} = q_{\vec \chi}\circ \vec{\psi}^{-1}$ et 
$$ 
\begin{array}{rcl}
\nabla q_{\vec X} &=& \left( \partial _{X}q_{\vec X}, \partial_t q_{\vec X}\right) \\
&=&\nabla q_{\vec \chi}\cdot \nabla \vec{\psi}^{-1}; \\
 \partial _{X}q_{\vec X} &=&  \partial _{\chi}q_{\vec \chi}\cdot\partial _{X}  \vec{\psi}^{-1}, \\
 \partial _{t}q_{\vec X} &=&  \partial _{\chi}q_{\vec \chi}\cdot\vec{w}+\partial_t q_{\vec \chi}.
\end{array}
$$
%
et pat définition:
%
$$
\begin{array}{rcl}
\partial _{t}q_{\vec X} &=& \partial_t q_{\vec \chi} +  \partial _{\chi}q_{\vec \chi}\cdot\vec{w} \\
&=&\partial_t q_{\vec \chi} +  \partial _{x}q_{\vec x}\cdot\partial_{\vec \chi}\vec \phi_{\vec x}\cdot\vec{w} \\
&=&\partial_t q_{\vec \chi} +  \partial _{x}q_{\vec x}\cdot\vec{c}
\end{array}
$$
%
\section{Formulation faible}
%
L'EDP:							
\begin{equation}
\label{eq:EDP}
\left\{
\begin{array}{rcll}
\nthpartial{u}{t}{} - k\,\laplacien u & = & f,  & \forall x \in \overline{\Omega} = \Omega\cup\partial\Omega; \\
u &=& u_0, & \forall x \in \Gamma_0 \subset \partial \Omega;\\
\nthpartial{u}{\vec{n}}{} &=& g, & \forall x \in \Gamma_g = \partial \Omega\setminus\Gamma_0;\\
u(x,t=0)&=&0,& \forall x \in \overline{\Omega}.
\end{array}
\right.
\end{equation}
							%
avec $$u\in\spacename V \triangleq \left\{v\in\spacename{L}^2(\Omega);\, \nthpartial{v}{x}{}\in\spacename{L}^2(\Omega), \, v(x\in\Gamma_0)=u_0 \right\}.$$	

N.B. Les conditions frontières essentielle (de type Dirichlet) sont incluses dans la définition de l'espace solution; Les conditions frontières naturelles (de type Neumann) sont à traiter séparément (\cf~formulation faible du problème ci-dessous).
%
On introduit les fonctions test $$v\in\hat{\spacename V}\triangleq\left\{v\in\spacename{L}^2(\Omega);\, \nthpartial{v}{x}{}\in\spacename{L}^2(\Omega), \, v(x\in\Gamma_0)=0 \right\};$$ \ie~ $\hat{\spacename V}$ est identique à $\spacename V$ partout sauf sur le bord $\partial\Omega$ où les fonctions test sont nulles.
							%
On écrit la formulation faible de (\ref{eq:EDP}):
\begin{enumerate}
\item Multiplication par un membre quelconque de $\hat{\spacename V}$ et intégration sur le domaine $\Omega$:
 $$\int_\Omega\nthpartial{u}{t}{}\, v\,\d \Omega - \int_\Omega k\,\laplacien u\, v \d \Omega= \int_\Omega f\, v \d \Omega. $$
 \item Intégration par partie, en exploitant la définition de l'espace des fonctions test: 
 $$\int_\Omega\nthpartial{u}{t}{}\, v\,\d \Omega - \underbrace{\left[\nabla u\,k \,v \right]_{\partial\Omega}}_{=0} + \int_\Omega k\,\nabla u\, \nabla v \d \Omega= \int_\Omega f\, v \d \Omega. $$
 \item Finalement
 \begin{equation}
 \label{eq:formulation_faible}
\int_\Omega\nthpartial{u}{t}{}\, v\,\d \Omega + \int_\Omega k\,\nabla u\, \nabla v \d \Omega= \int_\Omega f\, v \d \Omega.
 \end{equation}
\end{enumerate}
							%
							%
							%
\section{Discrétisation}						
							%
On considère un sous espace discret de l'espace des fonctions solutions ($h$ réfère au pas de discrétisation spatiale):
$$  \spacename V_h \subset \spacename V.$$
%
Une base de $\spacename V_h$ est $\{\phi_i\}_{1\leq i\leq \nnodes}$ (typiquement, les $\phi_i$ sont fournis par les fonctions de forme dans la méthode des éléments finis \cite[sec. 1.4]{ern2013theory} pour la résolution du problème variationnel (\ref{eq:formulation_faible})). 
%
Identiquement, une base de $\hat{\spacename V_h}$ est $\{\hat{N_i}\}_{1\leq i\leq \nnodes}$.
							%
							%
							%
On note $u_h \simeq u$ une solution approchée, que l'on décompose sur la base $\{\phi_i\}_{1\leq i\leq \nnodes}$ comme:
$$u_h(x,t)\triangleq \alpha_i(t)\,\phi_i(x).$$
							%
Le problème variationnel s'écrit alors 
							%
							%
							%
$$
\nthpartial{\alpha_i}{t}{}\,\int_\Omega \phi_i \, v\,\d \Omega + \alpha_i\,\int_\Omega k\,\nabla \phi_i\, \nabla v \d \Omega = \int_\Omega f\, v \d \Omega;\quad\forall v\in\hat{\spacename V}_h.
$$
							%
							%
Et en particulier pour $v=\hat{\phi}_j$:
							%
$$
\nthpartial{\alpha_i}{t}{}\,\int_\Omega \phi_i \, \hat{\phi}_j\,\d \Omega + \alpha_i\,\int_\Omega k\,\nabla \phi_i\, \nabla \hat{\phi}_j \d \Omega = \int_\Omega f\, \hat{\phi}_j \d \Omega;\quad 1\leq j \leq \nnodes,
$$
							%
soit 
\begin{equation}
\label{eq:Systeme_dynamique_discret}
\mat{A}\cdot\nthpartial{\vec{\alpha}}{t}{} + \mat{B}\cdot\vec{\alpha} = \vec{f},
\end{equation}
%
avec $A_{j,i} = \left( \phi_i \vert \hat{\phi}_j\right)$, $B_{j,i} = k\, \left( \nabla \phi_i \vert \nabla \hat{\phi}_j \right)$ et $f_j = \left( f \vert \nabla \hat{\phi}_j \right).$
%
\section{Base POD - Méthode "Snapshots"}
%
On dispose de $\nt$ mesures du champ de vitesse instantané $\{\bar u(x,t_k)\}_{1\leq k \leq \nt}$. On cherche une base orthonormale de $V$, de dimension finie, qui caractérise au mieux les $\nt$ clichés \textit{en moyenne} (ici temporelle) et \textit{au sens des moindres carrés}. C'est à dire que l'on cherche la séquence des $\{\varphi_i\}_{1\leq i \leq \npod}$ telle que 
$$\frac{\langle \left( u \vert \varphi \right) \rangle}{ \left( \varphi \vert \varphi \right)} = \underset{\psi \in\spacename L ^2}{\operatorname{max}}\left( \frac{\langle \left( u \vert \psi \right) \rangle}{ \left( \psi \vert \psi \right)} \right),$$
où l'opérateur de moyenne $\langle \cdot \rangle$ est à préciser et $(\cdot\vert \cdot)$ dénote le produit scalaire standard sur $\spacename L^2$.
%
On montre (\cf~\cite[sec. 4.1 et eq. (36)]{cordier2003proper}) que ce problème de maximisation est équivalent au problème au valeurs propres
\begin{equation}
\label{eq:probleme_aux_valeurs_propres}
\int_\Omega\operatorname R(x, x^\prime)\,\varphi(x^\prime)\,\d x^\prime = \lambda\,\varphi(x),
\end{equation}
où $\operatorname R$ dénote l'opérateur de corrélation spatiale. Dans le cas de la POD par Snapshots, la moyenne correspond à la moyenne d'ensemble sur les $\nt$ clichés $\langle \alpha(x, t) \rangle \triangleq \frac{1}{\nt}\sum_{k=1}^{\nt}\alpha(x,t_k)$, et \textbf{sous les hypothèses de stationnarité et d'ergodicité}, on a~\cite[p. 32]{cordier2003proper}:
\begin{equation}
R(x, x^\prime) = \frac{1}{\nt}\,\sum_{i=1}^{\nt} u(x,t_i)\otimes u^*(x^\prime,t_i)
\end{equation}
%
On fait alors l'hypothèse que les $\varphi_i$ s'écrivent comme combinaison linéaire des $\nt$ clichés de la solution:\begin{equation}
\label{eq:ersatz_base_pod}
\varphi_j = \gamma_{j,k}\,\bar{u}(x, t_k).
\end{equation}
%
En injectant l'ersatz (\ref{eq:ersatz_base_pod}) dans le problème aux valeurs propres (\ref{eq:probleme_aux_valeurs_propres}), on obtient~\cite[eq. (37)]{cordier2003proper}:
\begin{equation}
\label{eq:probleme_aux_valeurs_propres_sur_les_gammas_ij}
\frac{1}{\nt}\,\sum_{k=1}^{\nt}\big(u(x^\prime,t_k) \vert u(x^\prime,t_i)\big)\gamma_{j,k} = \lambda_j\,\gamma_{j,i},\quad 1\leq i \leq \nt,
\end{equation}
\ie
\begin{equation}
\mat{C}\,\gamma_j = \lambda_j\,\gamma_j,
\end{equation}
où $\gamma_j = (\gamma_{j,1},\cdots,\gamma_{j,\nt})^\T$ et 
$$ 
\begin{array}{rcl}
C_{k,i} & = & \frac{1}{\nt}\,\int_\Omega u^*(x, t_k) \, u(x, t_i)\,\d \Omega \\
& \simeq & \frac{1}{\nt}\,\sum_{n=1}^{\nnodes} u^*(x_n, t_k) \, u(x_n, t_i).
\end{array}
$$
\section{Modèle réduit}
%
\subsection{Structure de données}
%
On dispose d'une matrice $\widetilde{\mat{D}}$ représentant des données simulées:
\begin{equation}
\widetilde{\mat{D}} = \left( 
\begin{array}{ccc}
\vec{u}(\vec{x}_1; t_1)& \cdots & \vec{u}(\vec{x}_1; t_\nt) \\
\vdots & & \vdots\\
\vec{u}(\vec{x}_\nx; t_1)& \cdots & \vec{u}(\vec{x}_\nx; t_\nt) \\
\end{array}
\right) \in \RR^{\nx \times \nt}
\end{equation}
avec 
\begin{itemize}
\item la séquence de points du domaine spatial $(\vec x_i)_{1\leq i \leq \nx}$, $\vec x _i \in \Omega$,
\item la séquence ordonnée d'instants $(t_j)_{1\leq j \leq  \nt }$, $t_j\in \spacename T$, $t_j< t_{j+1}$.
\end{itemize}
%
Le problème est de restructurer ces données afin de définir des équivalents discrets aux opérateurs \emph{produit scalaire} et \emph{gradients}.
%
On choisit de représenter ces données par un tenseur $\Dtens$ d'ordre $\nc$ exprimé en coordonnées sur une base $\{\vec{e}_c\}_{1\leq c\leq \nc}$ avec $\nc$ le nombre de composantes spatiales.
%
On associe au domaine $\Omega\subset\RR^{\nc}$ le plus petit domaine parallélépipédique $\overline{\Omega} = \left\{\vec x \in \prod_{c=1}^\nc [x^{\labelname{min}}_c,\, x^{\labelname{max}}_c] \subseteq \RR^{\nc} \right\}$ qui contient $\Omega$.

\subsection{Opérateurs discrets}
%
On cherche à définir les opérateurs gradient, et produit scalaire nécessaires à la formulation du modèle réduit.
%
\paragraph{Gradient discret}
%
Par définition du gradient




\subsection{Projection sur la base POD}

On suppose les $\{\phi_i\}_{1\leq i \leq N_{\labelname{POD}}}$ orthonormés\footnote{Si la base n'est que orthogonale, on normalise chaque élément: $\phi_i\leftarrow (\phi_i \vert \phi_i)^{-1}\,\phi_i$; $1\leq i\leq N_{\labelname{POD}}$.}.
%
Dans toute la suite, le produit scalaire considéré correspond au produit scalaire discret:
%
\begin{equation}
\label{eq:produit_scalaire_discret}
\left.
\begin{array}{l}
\vec{u} = (u_i)_{1 \leq i \leq \nnodes}\\
\vec{v} = (v_j)_{1 \leq j \leq \nnodes}
\end{array}
\right\}
\rightarrow
(\vec u \vert \vec v) \triangleq u_i\,v_i.
\end{equation}
%
%
\subsection{formulation faible sur la base POD}
On réécrit le problème (\ref{eq:EDP}) pour la solution approchée sur la base POD $\{\varphi_j(x)\}_{1\leq j \leq N_{\labelname{POD}}\leq N_{\labelname{t}}}$:
$$\tilde u(x,t) = \beta_j(t)\,\varphi_j(x), $$
c'est à dire
 $$\nthpartial{\beta_j}{t}{} \underbrace{\int_\Omega\varphi_j\, \varphi_i\,\d \Omega}_{\overline{A}_{i,j}} -\beta_j\, \underbrace{\left[k \,\nabla \varphi_j\,\varphi_i \right]_{\partial\Omega}}_{\overline{C}_{i,j}} + \beta_j\,\underbrace{\int_\Omega k\,\nabla  \varphi_j\, \nabla \varphi_i \d \Omega}_{\overline{B}_{i,j}}= \underbrace{\int_\Omega f\, \varphi_i \d \Omega}_{\overline{f}_i},\quad 1\leq i \leq N_{\labelname{POD}}, $$
ou encore, avec $\beta=(\beta_j)_{1\leq j\leq N_{\labelname{POD}}}$:
$$\overline{\mat A}\,\nthpartial{\beta}{t}{} + (\overline{\mat B}-\overline{\mat C})\,\beta = \vec{\overline{f}}.$$
%===============================================================
%===============================================================
							%
\bibliographystyle{plain}
\bibliography{/home/afalaize/Documents/BIBLIO/JabRefDataBase}
							%
%===============================================================
%===============================================================
							%
							%
							%
\end{document}
